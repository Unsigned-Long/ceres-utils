\documentclass[12pt, onecolumn]{article}

% 引入相关的包
\usepackage{amsmath, listings, fontspec, geometry, graphicx, ctex, color, subfigure, amsfonts, amssymb}
\usepackage{multirow}
\usepackage[table,xcdraw]{xcolor}
\usepackage[ruled]{algorithm2e}
\usepackage[hidelinks]{hyperref}
\hypersetup{
	colorlinks=true,
	linkcolor=red,
	citecolor=red,
}

% 设定页面的尺寸和比例
\geometry{left = 1.5cm, right = 1.5cm, top = 1.5cm, bottom = 1.5cm}

% 设定两栏之间的间距
\setlength\columnsep{1cm}

% 设定字体,为代码的插入作准备
\newfontfamily\ubuntu{Ubuntu Mono}
\newfontfamily\consolas{Consolas}

% 头部信息
\title{\normf{基于连续时间的LiDAR/Camera/IMU的时空标定方法}}
\author{\normf{陈烁龙}}
\date{\normf{\today}}

% 代码块的风格设定
\lstset{
	language=C++,
	basicstyle=\scriptsize\ubuntu,
	keywordstyle=\textbf,
	stringstyle=\itshape,
	commentstyle=\itshape,
	numberstyle=\scriptsize\ubuntu,
	showstringspaces=false,
	numbers=left,
	numbersep=8pt,
	tabsize=2,
	frame=single,
	framerule=1pt,
	columns=fullflexible,
	breaklines,
	frame=shadowbox, 
	backgroundcolor=\color[rgb]{0.97,0.97,0.97}
}

% 字体族的定义
% \fangsong \songti \heiti \kaishu
\newcommand\normf{\fangsong}
\newcommand\boldf{\heiti}
\newcommand\keywords[1]{\boldf{关键词:} \normf #1}

\newcommand\liehat[1]{\left[ #1 \right]_\times}
\newcommand\lievee[1]{\left[ #1 \right]^\vee}
\newcommand\liehatvee[1]{\left[ #1 \right]^\vee_\times}

\newcommand\mlcomment[1]{\iffalse #1 \fi}
%\newcommand\mlcomment[1]{ #1 }

\begin{document}
	
	% 插入头部信息
	\maketitle
	% 换页
	\thispagestyle{empty}
	\clearpage
	
	% 插入目录、图、表并换页
	\pagenumbering{roman}
	\tableofcontents
	\newpage
	\listoffigures
	\newpage
	\listoftables
	% 罗马字母形式的页码
	
	\clearpage
	% 从该页开始计数
	\setcounter{page}{1}
	% 阿拉伯数字形式的页码
	\pagenumbering{arabic}
	
	\section{\normf{Sphere Manifold}}
	\normf
	Ceres中的$SphereManifold$提供了球面流形,球面上向量的范数保持不变。 这种情况经常出现,比如与IMU相关的多源融合定位中重力向量的估计。
	
	\subsection{\normf{Plus}}
	\begin{equation}
	\boldsymbol{x}\boxplus\delta\boldsymbol{x}=
	\left( 
	\sin\left( \frac{\left\| \delta\boldsymbol{x}\right\|}{2} \right) \cdot
	\frac{\delta\boldsymbol{x}^T}{\left\| \delta\boldsymbol{x}\right\| }
	\quad
	\cos\left(  \frac{\left\| \delta\boldsymbol{x}\right\|}{2} \right) 
	\right)^T  \cdot\left\| \boldsymbol{x}\right\| 
	\end{equation}
	
	\subsection{\normf{Plus Jacobian}}
	\begin{equation}
	\boldsymbol{J}_{c(i)}=	-\frac{\left\| \boldsymbol{x}\right\| }{2}\cdot\beta\cdot\boldsymbol{v}_{(i)}\cdot\boldsymbol{v}+\frac{\left\| \boldsymbol{x}\right\|}{2}
	\end{equation}
	也即:
	\begin{equation}
	\boldsymbol{J}=\frac{1}{2}\cdot\boldsymbol{H}_{c(0:d)}
	\end{equation}
	
	\subsection{\normf{Minus}}
	\begin{equation}
	\boldsymbol{x}\boxminus\boldsymbol{y}=
	2\cdot\mathrm{atan}\left(\frac{\left\| \boldsymbol{y}^\prime \right\| }{\boldsymbol{y}^\prime_{(-1)}}\right) 
	\cdot\frac{1}{\left\| \boldsymbol{y}^\prime \right\| }\cdot 
	\boldsymbol{y}^\prime_{(0:d)}
	\quad\mathrm{with}\quad\boldsymbol{y}^\prime=\boldsymbol{H}\cdot\boldsymbol{y} 
	\end{equation}
	
	\subsection{\normf{Minus Jacobian}}
	\begin{equation}
	\boldsymbol{J}_{c(i)}=-\frac{2}{\left\| \boldsymbol{x}\right\|}\cdot\beta\cdot\boldsymbol{v}_{(i)}\cdot\boldsymbol{v}+
	\frac{2}{\left\| \boldsymbol{x}\right\|}
	\end{equation}
	也即:
	\begin{equation}
	\boldsymbol{J}=2\cdot\boldsymbol{H}_{c(0:d)}
	\end{equation}
	
	\subsection{\normf{Householder Vector}}
	对于输入的向量$\boldsymbol{x}$,使用其最后一个元素作为主元(而不是第一个)。$\boldsymbol{v}$的householder变换:
	\begin{equation}
	\boldsymbol{H}=\boldsymbol{I}-\beta\cdot\boldsymbol{v}\cdot\boldsymbol{v}^T
	\end{equation}
	该householder变换矩阵作用于向量时,会以与$\boldsymbol{v}$正交的平面作为“镜面”,将向量进行反射:
	\begin{equation}
	\boldsymbol{H}\cdot\boldsymbol{x}=\boldsymbol{x}^\prime
	\end{equation}
	比如,直接作用于$\boldsymbol{v}$,会得到$-\boldsymbol{v}$;作用于与$\boldsymbol{v}$正交的向量$\boldsymbol{u},\boldsymbol{u}^T\cdot\boldsymbol{v}=0$,则仍然保持$\boldsymbol{u}$不变。
	
\end{document}

